%!TEX root = GRoutes.tex
%%%%%%%%%%%%%%%%%%%%%%%%%%%%%%%%%%%%%%%%%%%%%%%%%%%%%%%%%%%%%%%%%%%%%%%
\chapter{Gráfok - alapfogalmak}\label{ch:ALAP}
%%%%%%%%%%%%%%%%%%%%%%%%%%%%%%%%%%%%%%%%%%%%%%%%%%%%%%%%%%%%%%%%%%%%%%%

\begin{osszefoglal}
	Ebben a fejezetben a későbbiekben használt fogalmakat fogom definiálni.
	
\end{osszefoglal}

Egy gráf \(G = (V(G),E(G)\) vagy \(G = (V,E))\) két véges halmazból áll. \(V(G)\) vagy \(V\), egy nem üres halmaz, melynek az elemeit csúcsoknak nevezzük, ezek alkotják a gráf csúcsait. \(E(G)\) vagy \(E\), egy halmaz, melynek elemeit éleknek nevezzük, ezek alkotják a gráf éleit, úgy, hogy minden \(e \in E\) élet meghatároz egy rendezetlen csúcs-pár \((u,v)\), melyeket \(e\) csúcsainak nevezünk.

\begin{description}
	\setlength{\itemsep}{0.04mm}
	\item[rend] -- definíció szerint a \(G\) gráf rendje \(|V| = n\)
	\item[méret] -- definíció szerint a \(G\) gráf mérete \(|E| = m\)
	\item[hurokél] -- olyan él, melynek mindkét végpontja megegyezik
	\item[többszörös él] -- ha két csúcsot több él köt össze, akkor ezeket többszörös, vagy párhuzamos éleknek nevezzük
	\item[egyszerű gráf] -- azon gráf, amely nem tartalmaz sem hurokélt, sem többszörös éleket
	\item[teljes gráf] -- egy egyszerű gráf, amelynek minden különböző csúcs-párját összeköti egy él. Egy teljes gráf \(n(n-1)/2\) éllel rendelkezik. Ha a teljes gráf csúcsai \(v_1, v_2, \ldots, v_n\), akkor az élek halmaza megadható a következőképpen:
	\begin{equation}
	E = \{(v_i,v_j): v_i \neq v_j;\;\;\;i,j = 1,2,3\dots, n\}
	\end{equation}
	\item[irányított gráf] -- olyan gráf, ahol az éleket rendezett \((u,v)\) csúcspárok határozzák meg (számít, hogy melyik a kezdő- és végpont)
	\item[részgráf] -- Legyen \(H\) egy gráf, csúcsainak halmaza \(V(H)\), éleinek halmaza \(E(H)\), hasonlóan \(G\) egy gráf, csúcsainak halmaza \(V(G)\) és éleinek halmaza \(E(G)\). \(H\) részgráfja \(G\)-nek, ha \(V(H) \subseteq V(G)\) és \(E(H) \subseteq E(G)\).
	\item[séta] -- a csúcsok és élek váltakozó véges sorozata, mely csúccsal kezdődik és csúccsal végződik, valamint minden csúcsot egy vele szomszédos él követ és fordítva
	\item[vonal] -- az a séta, melyben az élek nem ismétlődnek.
	\item[út] -- az a séta, melyben a csúcspontok nem ismétlődnek.
	\item[kör] -- egy nem-triviális%
	\footnote{ %
		hossza nagyobb, mint 0
	}  %
 zárt vonal, amelynek a kezdő- és belső pontjai nem ismétlődnek.
	\item[Hamilton-út] -- egy \(G\) gráf azon útja, mely minden csúcsot magába foglal
\end{description}

