%!TEX root = GRoutes.tex
%%%%%%%%%%%%%%%%%%%%%%%%%%%%%%%%%%%%%%%%%%%%%%%%%%%%%%%%%%%%%%%%%%%%%%%
\chapter{Gráfok - alapfogalmak}\label{ch:ALAP}
%%%%%%%%%%%%%%%%%%%%%%%%%%%%%%%%%%%%%%%%%%%%%%%%%%%%%%%%%%%%%%%%%%%%%%%

\begin{osszefoglal}
	Ebben a fejezetben a későbbiekben használt fogalmakat fogom definiálni.
	
\end{osszefoglal}

Egy gráf \(G = (V(G),E(G)\) vagy \(G = (V,E))\) két véges halmazból áll. \(V(G)\) vagy \(V\), egy nem üres halmaz, melynek az elemeit csúcsoknak nevezzük, ezek alkotják a gráf csúcsait. \(E(G)\) vagy \(E\), egy halmaz, melynek elemeit éleknek nevezzük, ezek alkotják a gráf éleit, úgy, hogy minden \(e \in E\) élet meghatároz egy csúcs-pár \((u,v)\), melyeket \(e\) csúcsainak nevezünk.

\begin{description}
	\setlength{\itemsep}{0.04mm}
	\item[rend] -- definíció szerint a \(G\) gráf rendje \(|V| = n\)
	\item[méret] -- definíció szerint a \(G\) gráf mérete \(|E| = m\)
\end{description}

