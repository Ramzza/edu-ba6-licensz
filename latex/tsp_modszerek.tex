%!TEX root = GRoutes.tex
%%%%%%%%%%%%%%%%%%%%%%%%%%%%%%%%%%%%%%%%%%%%%%%%%%%%%%%%%%%%%%%%%%%%%%%
\chapter{Az utazó ügynök problémája (TSP)}\label{ch:ALAP}
%%%%%%%%%%%%%%%%%%%%%%%%%%%%%%%%%%%%%%%%%%%%%%%%%%%%%%%%%%%%%%%%%%%%%%%


Az utazó ügynök problémája magába foglalja az Hamilton-kör létezésnek a problémáját is. Adott valamennyi város, amelyeket az utazó ügynök úgy kell meglátogasson, hogy mindegyik városba egyszer és csakis egyszer menjen be, valamint érjen vissza a kiindulási pontba.

%%%%%%%%%%%%%%%%%%%%%%%%%%%%%%%%%%%%%%%%%%%%%%%%%%%%%%%%%%%%%%%%%%%%%%%
\section{A probléma komplexitása}\label{sec:ALAP:adatelem}

\begin{description}
	\setlength{\itemsep}{0.04mm}
	\item[P] -- azokat a problémákat foglalja magába, melyeket egy determinisztikus Turing-gép polinomiális időben képes megoldani
	\item[NP] -- azokat a problémákat foglalja magába, melyeket egy nem-determinisztikus Turing gép polinomiális időben képes megoldani. Ezeknek a megoldását polinomiális időn belül le lehet ellenőrizni determinisztikus Turing-géppel.
	\item[NP-nehéz] -- egy probléma amely "legalább olyan nehéz, mint a legnehezebb probléma az NP-ben"
	\item[NP-teljes] -- egy probléma NP-teljes ha úgy az NP, mint az NP-nehéz halamznak is eleme, így tehát ezek a legnehezebb komputacionális problémák
\end{description}


Az utazó ügynök problémája magába foglalja a Hamilton kör problémáját is, ami NP-teljes, így tehát a TSP%
\footnote{ %
	travelling talesman problem - az utazó ügynök problémája
}  %
 is NP-teljes.
%%%%%%%%%%%%%%%%%%%%%%%%%%%%%%%%%%%%%%%%%%%%%%%%%%%%%%%%%%%%%%%%%%%%%%%
\section{Egzakt megoldások}\label{sec:ALAP:adatelem}

Az egzatk algoritmusok minden esetben az optimális megoldást térítik vissza, komplexitásuk azonban exponenticális. Így a csomópontok növekedésével a futási sebesség exponenciálisan fog nőni. Kutatási szemptontot leszámítva, nem lehetne ezeket az algoritmusokat átlagos felhasználói környezetben alkalmazni, ugyanis már 20 csomópont után nem lehet kivárni a megoldást. 

Egzakt algoritmusok: a brute force keresés, Held-Karp alrogitmus, valamint a Concorde.

%%%%%%%%%%%%%%%%%%%%%%%%%%%%%%%%%%%%%%%%%%%%%%%%%%%%%%%%%%%%%%%%%%%%%%%
\section{Approximációk}\label{sec:ALAP:adatelem}

Mivel az egzakt algoritmusokat csak igazán kevés használati esetben lehet alkalmazni, ezért a mohó algoritmusok mellett is dönthetünk. Ezek polinomiális idő alatt elvégzik a feladatod, azonban nem mindig az optimumot térítik vissza. Az eltérés mértéke függhet azonban az implemetációtól.

Mohó algoritmusok: legközelebbi szomszéd algoritmus és a Chained Lin-Kernighan.

%%%%%%%%%%%%%%%%%%%%%%%%%%%%%%%%%%%%%%%%%%%%%%%%%%%%%%%%%%%%%%%%%%%%%%%
\section{Aszimmetrikus TSP}\label{sec:ALAP:adatelem}

Amennyiben irányított gráfokkal dolgozunk, abban az esetben az aszimmetrikus TSP-ről beszélünk. Nem minden TSP algoritmus működik ATSP-re is, ezért előfordulhat, hogy át kell alakítanunk a gráfot, visszavezetve a TSP-re. A visszavezetés után, \(n\) csomópontból álló ATSP egy \(2n\) csomópontból álló TSP-t fog eredményezni.