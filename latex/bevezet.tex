%!TEX root = minta_dolgozat.tex
%%%%%%%%%%%%%%%%%%%%%%%%%%%%%%%%%%%%%%%%%%%%%%%%%%%%%%%%%%%%%%%%%%%%%%%
\chapter{Bevezetés}\label{ch:ALAP}
%%%%%%%%%%%%%%%%%%%%%%%%%%%%%%%%%%%%%%%%%%%%%%%%%%%%%%%%%%%%%%%%%%%%%%%

Dolgozatom témája az "utazó ügynök" problémájának (TSP) a gyakorlatban történő alkalmazása. Az utazó ügynök problémája egy komputacionális optimalizálási probléma, amely az 1930-as évek óta nagyon intenzíven foglalkoztatja a tudományos világot. Egyszerűen megfogalmazva: adott valamennyi város, valamint az köztük levő távolságok. Melyik a legrövidebb lehetséges útvonal, ami egyszer érinti az összes várost, majd visszaérkezik a kiinduló pontba? Jelenleg nem létezik olyan polinomiális komplexitással rendelkező algoritmus, amely erre a problémára egzakt megoldást nyújtana.

Az utazó ügynök problémájának példázásának céljából egy android alkalmazást készítettem, amelyben különböző helyekre (csomópontokra) rákeresve eredményként egy olyan útvonal rajzolódik ki, amely az összes helyet érinti, valamint a legrövidebb is egyúttal. A felhasználó két utazási mód közül választhat: gyaloglás, autó. Az alkalmazás felhasználói célterülete a turistaútvonalak tervezői, a csomagkihordó szolgáltatást biztosító cégek, valamint a különböző eladási szakterülettel foglalkozó vállalkozások, ahol oda kell utazni az ügyfélhez.

Az első fejezetben a gráfelméleti alapfogalmakat fogom definiálni, részletezni. Ezek a későbbiekben fontosak lesznek, hogy egzakt módon taglalhassuk a témát. A következő fejezetben a különböző gráfelméleti problémákat, feladatokat fogom taglalni, majd ezt követően egy külön fejezetet szentelek az általam kiválasztott problémára, az "utazó ügynök" problémájára. Ebben a fejezetben meg kell magyaráznom a különböző komplexitási alapfogalmakat, hogy rávilágítsak a téma komputacionális nehézségeire. Arról is itt fog szó esni, hogy milyen módszerekkel közelítjük meg a problémát. A későbbiekben nagyító alá helyezem a különböző algoritmusokat az előbbiekben emlitett megoldási stratégiák szerint osztályozva. Bizonyos algoritmusok csak szimmetrikus TSP esetén adnak helyes eredményt, így az asszimetrikus TSP szimmetrikusra való visszavezetését is tárgyalni fogom.

A dolgozat gyakorlati része a projekt során felhasznált technológiák bemutatásával fog kezdődni. Ez fontos lehet azok számára, akik alapjaiban jobban meg szeretnék érteni az applikációt, vagy egy hasonló alkalmazás elkészítéséhez szeretnéknek hasznos ismereteket elsajátítani. A technológiák bemutatása után maga az alkalmazás nagyító alá helyezése következik. Ebben a részben fogok írni az applikáció különböző komponenseiről, arról, hogy ezek milyen kapcsolatban vannak egymással. Ezt osztálydiagrammal is fogom illusztálni. Itt fogok részletesebben írni az adatbázisról, az autentikációról, a Maps API-ról, az alkalmazás funkcionalitásairól, hogy milyen esetben melyik algoritmust alkalmazom és miért kell egyáltalán különböző esetekben más-más algoritmust használni, valamint a nehézségekről, melyekbe munkám során ütköztem, és ezek megoldásairól.

Végül levonom a következtetéseket a munkám során szerzett tapasztalatokból, valamint javaslatokat teszek az alkalmazás jövőbeli továbbfejlesztési lehetőségeire.
