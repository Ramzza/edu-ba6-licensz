%!TEX root = GRoutes.tex
%%%%%%%%%%%%%%%%%%%%%%%%%%%%%%%%%%%%%%%%%%%%%%%%%%%%%%%%%%%%%%%%%%%%%%%
\chapter{Bevezetés}
%%%%%%%%%%%%%%%%%%%%%%%%%%%%%%%%%%%%%%%%%%%%%%%%%%%%%%%%%%%%%%%%%%%%%%%

Dolgozatom témája az "utazó ügynök" problémájának (TSP)%
\footnote{ %
	Travelling salesman problem
}  %
 a gyakorlatban történő alkalmazása. Az utazó ügynök problémája egy komputacionális optimalizálási probléma, amely az 1930-as évek óta nagyon intenzíven foglalkoztatja a tudományos világot. Egyszerűen megfogalmazva: adott valamennyi város, valamint a köztük levő távolságok. Melyik a legrövidebb lehetséges útvonal, ami egyszer érinti az összes várost, majd visszaérkezik a kiinduló pontba? Jelenleg nem létezik olyan polinomiális komplexitással rendelkező algoritmus, amely erre a problémára egzakt megoldást nyújtana.

Az utazó ügynök problémájának példázásának céljából egy android alkalmazást készítettem, amelyben különböző helyekre (csomópontokra) rákeresve, eredményként kirajzolja az optimális (legrövidebb) útvonalat, amely érinti az összes csomópontot. A felhasználó két utazási mód közül választhat: gyaloglás, illetve vezetés. Az alkalmazás felhasználói célközönsége a turistaútvonalak tervezői, a csomagkihordó szolgáltatást biztosító cégek, valamint a különböző eladási szakterülettel foglalkozó vállalkozások, ahol oda kell utazni az ügyfélhez. Létezik egy grafikus felhasználói felülettel rendelkező program%
\footnote{ %
	Concorde TSP Solver
}  %
 Windows operációs rendszerre, amely megoldja az utazó ügynök problémáját, azonban ez egy általános problémmegoldó szoftver, nem minden esetben felel meg a felhasználói igényeknek. A GRoutes előnye ezzel az alkalmazással szemben az, hogy egy konkrét élethelyzetre próbál megoldást találni, valamint a felhasználó az okostelefonján veheti igénybe a szolgáltatást, bármikor, bárhol, és könnyedén tudja menedzselni a kereséseit, kedvenc útvonalait, elmentett helyszíneit, illetve a navigációra is van lehetőség.

Az első fejezetben a gráfelméleti alapfogalmakat fogjuk definiálni, részletezni. A továbbiakban a különböző gráfelméleti problémákat, feladatokat taglaljuk, majd ezt követően egy külön fejezetet szentelünk a dolgozat témájára, az "utazó ügynök" problémájára. Ebben a fejezetben meg kell magyaráznunk a különböző komplexitási alapfogalmakat, hogy rávilágítsunk a téma komputacionális nehézségeire. Arról is itt fog szó esni, hogy milyen módszerekkel közelítjük meg a problémát. A későbbiekben nagyító alá helyezzük a különböző algoritmusokat az előbbiekben említett megoldási stratégiák szerint osztályozva. Bizonyos algoritmusok csak szimmetrikus TSP esetén adnak helyes eredményt, így az asszimetrikus TSP szimmetrikusra való visszavezetését is tárgyalni fogjuk.

A dolgozat gyakorlati része a projekt során felhasznált technológiák bemutatásával fog kezdődni. Ez fontos lehet azok számára, akik alapjaiban jobban meg szeretnék érteni az applikációt, vagy egy hasonló alkalmazás elkészítéséhez szeretnéknek hasznos ismereteket elsajátítani. A technológiák bemutatása után maga az alkalmazás áttekintése következik. Ebben a részben taglaljuk az applikáció különböző komponenseit, hogy ezek milyen kapcsolatban vannak egymással. Ezt osztálydiagrammal is illusztráljuk. Itt lesz szó részletesebben az adatbázisról, az autentikációról, a Maps API-ról%
\footnote{ %
	Application Programming Interface
}  %
, az alkalmazás funkcionalitásairól, hogy milyen esetben melyik algoritmus van alkalmazva és miért kell egyáltalán különböző esetekben más-más algoritmust használni, valamint a nehézségekről, melyekbe munkám során ütköztem, és ezek megoldásairól.

Végül levonjuk a következtetéseket a munkám során szerzett tapasztalatokból, valamint javaslatokat teszünk az alkalmazás jövőbeli továbbfejlesztési lehetőségeire.
