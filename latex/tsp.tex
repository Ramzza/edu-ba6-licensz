%!TEX root = GRoutes.tex
%%%%%%%%%%%%%%%%%%%%%%%%%%%%%%%%%%%%%%%%%%%%%%%%%%%%%%%%%%%%%%%%%%%%%%%
\chapter{Az utazó ügynök problémája (TSP)}\label{ch:ALAP}
%%%%%%%%%%%%%%%%%%%%%%%%%%%%%%%%%%%%%%%%%%%%%%%%%%%%%%%%%%%%%%%%%%%%%%%


Az utazó ügynök problémája magába foglalja a Hamilton-kör létezésnek a problémáját is. Adott valamennyi város, amelyeket az utazó ügynök úgy kell meglátogasson, hogy mindegyik városba egyszer és csakis egyszer menjen be, valamint érjen vissza a kiindulási pontba.

%%%%%%%%%%%%%%%%%%%%%%%%%%%%%%%%%%%%%%%%%%%%%%%%%%%%%%%%%%%%%%%%%%%%%%%
\section{A probléma komplexitása}\label{sec:ALAP:adatelem}

\begin{description}
	\setlength{\itemsep}{0.04mm}
	\item[P] -- azokat a problémákat foglalja magába, melyeket egy determinisztikus Turing-gép polinomiális időben képes megoldani
	\item[NP] -- azokat a problémákat foglalja magába, melyeket egy nem-determinisztikus Turing gép polinomiális időben képes megoldani. Ezeknek a megoldását polinomiális időn belül le lehet ellenőrizni determinisztikus Turing-géppel.
	\item[NP-nehéz] -- egy probléma amely "legalább olyan nehéz, mint a legnehezebb probléma az NP-ben"
	\item[NP-teljes] -- egy probléma NP-teljes ha úgy az NP, mint az NP-nehéz halamznak is eleme, így tehát ezek a legnehezebb komputacionális problémák
\end{description}


Az utazó ügynök problémája magába foglalja a Hamilton kör problémáját is, ami NP-teljes, így tehát a TSP%
\footnote{ %
	travelling talesman problem - az utazó ügynök problémája
}  %
 is NP-teljes.

%%%%%%%%%%%%%%%%%%%%%%%%%%%%%%%%%%%%%%%%%%%%%%%%%%%%%%%%%%%%%%%%%%%%%%%
\section{Aszimmetrikus TSP}\label{sec:ALAP:adatelem}

Amennyiben irányított gráfokkal dolgozunk, abban az esetben az aszimmetrikus TSP-ről beszélünk. Nem minden TSP algoritmus működik ATSP-re is, ezért előfordulhat, hogy át kell alakítanunk a gráfot, visszavezetve a TSP-re. A visszavezetés után, \(n\) csomópontból álló ATSP egy \(2n\) csomópontból álló TSP-t fog eredményezni.

\section{Egzakt algoritmusok}\label{sec:ALAP:adatelem}

Az egzatk algoritmusok minden esetben az optimális megoldást térítik vissza, komplexitásuk azonban exponenticális. Így a csomópontok növekedésével a futási sebesség exponenciálisan fog nőni. Kutatási szemptontot leszámítva, nem lehetne ezeket az algoritmusokat átlagos felhasználói környezetben alkalmazni, ugyanis már 20 csomópont után nem lehet kivárni a megoldást. 

\subsection{Brute force}

A legegyszerűbb megközelítés, az összes permutációt kipróbálni, és kiválasztani ezek közül a legjobbat. Ennek a komplexitása azonban \(O(n!)\), így ezt már 20 városnál se lehet alkalmazni.


\subsection{Held-Karp algoritmus}

Az algoritmus komplexitása a legrosszabb esetben \(O(n^2*2^n)\).

Legyen \(S \subseteq {2, \dots, N}\) részhalmaza a városoknak, és \(c \in S\) úgy, hogy \(D(S,c)\) a minimális távolság kezdve az első várostól, meglátogatva az összes várost \(S\)-ből, majd visszaérve \(c\) városba.

ha \(S = {c}\), akkor \(D(S,c) = d_{1,c}\), különben:

\begin{equation}
D(S,c) = min_{x \in S-c}(D(S - c,x)+d_{x,c})
\end{equation}

Ezután a minimális út az összes város érintésével:

\begin{equation}
M = min_{c \in \{2, \dots, N\}}(D(\{2, \dots, N\}, c)+d_{c, 1})
\end{equation}

Egy \{\(n_1, \dots, n_N\)\} út minimális, ha teljesíti:

\begin{equation}
M = (D(\{2, \dots, N\}, n_N)+d_{n_N, 1})
\end{equation}

\subsection{Concorde}

A \(Concorde\) TSP megoldó egzakt megoldást nyújt az útazó ügynök problémájára. ANSI C-ben íródott és a "cutting plane" módszer segítségével iteratívan oldja meg a TSP lineáris programozási relaxációit. A felhasználói grafikus felület lehetőséget ad arra, hogy különböző heurisztikus algoritmusokkal számoljuk ki a megoldást. 

A \(Concorde\) segítségével megtalálták az optimális megoldást a TSPLIB%
\footnote{ %
	a TSP-re példa adatokat tartalmazó könyvtár
}  %
mind a 110 példájára, ahol a legnagyobb 85900 várost tartalmaz.

Az alábbi eredmények egy 2.8 GHz-es Intel Xeon processzor egy magjának használatával lettek kiszámolva, az \textit{ILOG CPLEX} lineáris programozási feladatmegoldó segítségével. Mindegyik adathalmaz 100 csomópontból áll.

\begin{table}[]
	\begin{tabular}{l|l}
		\textbf{Adathalmaz} & \textbf{Futási idő} (másodperc) \\
		\hline
		kroA100         & 0.31  \\
		kroB100         & 0.58 \\
		kroC100         & 0.30 \\   
		kroD100         & 0.33        
	\end{tabular}
\end{table}

%%%%%%%%%%%%%%%%%%%%%%%%%%%%%%%%%%%%%%%%%%%%%%%%%%%%%%%%%%%%%%%%%%%%%%%
\section{Heurisztikus algoritmusok}\label{sec:ALAP:adatelem}

\subsection{Greedy - legközelebbi szomszéd}

Mivel az egzakt algoritmusokat csak igazán kevés használati esetben lehet alkalmazni, ezért a mohó algoritmusok mellett is dönthetünk. Ezek polinomiális idő alatt elvégzik a feladatod, azonban nem mindig az optimumot térítik vissza. Az eltérés mértéke függhet azonban az implemetációtól.


\begin{description}
	\setlength{\itemsep}{0.04mm}
	\item[1. lépés] -- kezdjünk egy véletlenszerűen kiválasztott csomóponttal, melyet beállítunk aktuálisnak
	\item[2. lépés] -- keressük meg a legrövidebb élet, amely összeköti az aktuális csúcsot, és egy meg nem látogatott \(V\) csúcsot
	\item[3. lépés] -- beállítjuk \(V\)-t aktuális cs]csnak
	\item[4. lépés] -- megjelöljük, hogy már meglátogattuk \(V\)-t
	\item[5. lépés] -- ha minden csomópontot meglátogattunk, akkor algoritmus vége
	\item[6. lépés] -- menjünk a 2. lépéshez
\end{description}

\subsection{2-opt algoritmus}

A 2-opt algoritmus egy lokális kereső algoritmus, mely egy meglévő utat javít fel. Ezt az algoritmust elsőként Croes javasolta 1958-ban, az alapműveletet azonban már Flood is javasolta 1956-ban. Ez abból áll, hogy egy meglévő körútból kitörlünk két élet, úgy hogy ezeknek nincs közös csúcsúk, majd a csomópontokat újra összekötjük. Erre egy lehetőség van úgy, hogy ne az előző utat kapjuk. Amennyiben az újonnan kapott körút rövidebb, megtartjuk. Ezeket a cseréket minden lehetséges kombinációra elvégezzük. Az így kapot körutat 2-optimálisnak nevezzük.

\subsection{Lin-Kernighan heurisztika}

A 2-opt algoritmus általánosítása révén született meg az egyik leghatékonyabb approximációs algoritmus a szimmetrikus TSP megoldására, a Lin-Kernighan algoritmus. Egy körút k-optimális, ha nem lehetséges egy rövidebb körutat kapni \(k\) darab él, más \(k\) darab éllel való helyettesítés után. Minél nagyobb a \(k\) értéke, annál valószínűbb, hogy az algoritmus végrehajtása után az optimális megoldást kapjuk, ugyanakkor nagyobb adathalmazra gyorsan növekszik a \(k\) darab élcsere ellenőrzéséhez szükséges műveletek száma. Egy hátránya tehát a \(k-opt\) algoritmusoknak, hogy futás előtt meg kell adni a \(k\) értékét. Ezt a hátrányt igyekszik kiküszöbölni a Lin-Kernighan algoritmus, azáltal, hogy futása során változtatja \(k\)-nak az értékét.

Legyen \(T\) az aktuális körút. Minden iterációban, az algoritmus igyekszik találni két \(X = \{x_1,\dots,x_k\}\) és \(Y = \{y_1,\dots,y_k\}\) élekből álló halmazokat, melyekre igaz az, hogy a \(T\) körútból az \(X\)-ben található éleket az \(Y\)-ból vett élekkel helyettesítve egy jobb körutat kapunk. Ezeknek az éleknek a felcserélését \(k-opt\) lépésnek nevezzük. Ahhoz, hogy kellően hatékony algoritmust kapjunk, az \(X\) és \(Y\) halmazokhoz tartozó éleknek meg kell felelniük bizonyos kritériumoknak:

\begin{description}
	\setlength{\itemsep}{0.04mm}
	\item[(a) A szekvenciális csere kritériuma] -- \(x_i\)-nek, valamint \(y_i\)-nek rendelkezniük kell egy közös csúccsal, ugyanígy \(y_i\)-nek és \(x_{i+1}\)-nek is. Így tehát a \(x_1,y_1,x_2,y_2,\dots,x_k,y_k\) sorozat egy szomszédos élekből álló láncot alkot.
	\item[(b) A megvalósíthatósági kritérium] -- szükséges továbbá, hogy \(x_i = (t_\{2i-1\},t_\{2i\})\) úgy legyen kiválasztva, hogy ha a \(t_\{2i\}\)-t csatlakoztatjuk a \(t_1\)-hez, az így kapott gráf körút maradjon. Ez a kritérium az algoritmus futási idejének csökkentéséért, valamint a kódolás leegyszerűsítéséért lett bevonva.
	\item[(c) A pozitív nyereség kritériuma] -- az \(y_i\)-t úgy kell kiválasztani, hogy a nyereség a javaslot cserék után pozitív maradjon. Ez a feltétel kulcsfontosságú az algoritmus hatékonyságának szempontjából.
	\item[(d) A diszjunktivitás kritérium] -- az \(X\) és \(Y\) halmazoknak diszjunktnak kell lenniük
\end{description}