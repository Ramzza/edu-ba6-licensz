%!TEX root = GRoutes.tex
%%%%%%%%%%%%%%%%%%%%%%%%%%%%%%%%%%%%%%%%%%%%%%%%%%%%%%%%%%%%%%%%%%%%%%%
\chapter{GRoutes - alkalmazás bemutatása}\label{ch:ALAP}
%%%%%%%%%%%%%%%%%%%%%%%%%%%%%%%%%%%%%%%%%%%%%%%%%%%%%%%%%%%%%%%%%%%%%%%

\begin{osszefoglal}
	Ebben a fejezetben a GRoutes android applikációt fogom bemutatni úgy technikai, mint funkcionális szempontból.
	
\end{osszefoglal}

%%%%%%%%%%%%%%%%%%%%%%%%%%%%%%%%%%%%%%%%%%%%%%%%%%%%%%%%%%%%%%%%%%%%%%%
\section{Funkcionalitások}\label{sec:ALAP:adatelem}

\subsection{Bejelentkezés}

\begin{wrapfigure}{l}{0.4\linewidth}
	\centering
	\setlength{\abovecaptionskip}{0pt}
	\setlength{\belowcaptionskip}{0pt}
	\includegraphics[width=0.4\textwidth]{images/login}
	\caption{Bejelentkezési felület\label{fig:ALAP:sm2}}
\end{wrapfigure}

Az alkalmazás elindítása után egy bejelentkezési felület fogadja a felhasználót. Itt kell megadni az e-mail címet, valamint a jelszót amivel a felhasználó regisztrált. Amennyiben elfelejtette a jelszavát, az "elfelejtett jelszó" gomb megnyomásával lehetőség van egy jelszó-visszaállító e-mail kiküldésére. 

Ha egy új felhasználó szeretné igénybe venni az applikáció szolgáltatásait, akkor regisztrálnia kell. Ezt megteheti a regisztráció lapra való navigálás után, melyet a cím megérintésével, vagy a képernyőn az ujjának balra történő húzásával érheti el. Az e-mail cím és az új jelszó megadása után, a felhasználó egyből a főmenübe érkezik, mindeközben azonban egy aktivációs e-mailt is kap arra a címére, amivel regisztrált. A későbbiekben csak akkor fog tudni bejelentkezni, ha a kapott emailben az URL%
\footnote{ %
	Uniform Resource Locator, más néven webcím
}  %
-re klikkelve aktiválja a felhasználóját.

\subsection{Főmenü}

A főmenüből négy különböző oldalra navigálhatunk tovább: a keresési felület, a régebbi keresések megtekintése, a kedvencek menedzselése, valamint a beállítások. Az ötödik (csoportok) oldal egy jövőbeli funkcionalitás kifejlesztésére van fenntartva.

\subsection{Keresési felület}

\subsection{Térkép felület}

Ez a felulet akkor jelenik meg, amikor megerintjuk a terkep gombot egy helyszin vagy egy elmentett utvonal mellett, valamint a keresesi felulet eredmenye is itt jelenik meg.
Amennyiben egy helyintol erkezunk, megjelenik egy jelzo (marker) az adott koordinatan.
A csillag gomb segitsegevel hozzaadhatunk egy utvonalat a kedvencek koze, ahol kesobb modosithatjuk a nevet.

\subsection{Múltbeli keresések}

A keresési felület eredményéül kapott útvonalak megjelennek a múltbeli keresések menüpont alatt. Alapértelmezetten, a bejegyzések neve a keresési dátum és időpont lesz. Az "X" gombot megérintve törölhetünk egy bejegyzést, ezáltal az adatbázisból is törlődni fog. A "térkép" gomb megjeleníti a térkép felületet, ahol kirajzolódik az útvonal. Ez a felület a fentebb említett funkcionalitásokkal rendelkezik.

\subsection{Kedvencek}
A kedvencek felulet

Itt harom muveletet hajthatunk vegre, mindegyiknek megfelel egy gomb.

\subsection{Beállítások}

Itt ki tudjuk választani a limitet, hogy mennyi csomópont esetén, melyik algoritmust használja az alkalmazás. A megadott számnál kisebb vagy egyenlő számú csomópontok esetén a \(Concorde\) nevű egzakt megoldásokat nyújtó algoritmus fogja kiszámolni az ideális útvonalat. Ez az algoritmus a legpontosabb megoldásokat nyújtja, azonban az "utazó ügynök" problémájának komplexitása miatt, a csomópontok növekedésével, a végrehajtási idő is exponenciálisan növekszik. Amennyiben a felhasználó nagyon sok csomóponttal szeretne dolgozni, és fontosabb neki az, hogy belátható időn belül egy elfogadható megoldást kapjon, de nem probléma, ha nem a leghatékonyabb útvonal rajzolódik ki, akkor a megadott szám feletti mennyiségű csomópontok esetén, az applikáció egy úgynevezett \(greedy\)%
\footnote{ %
	mohó
}  %
 algoritmust fog használni. Ez nagyságrendekkel gyorsabb az exponenciálishoz viszonyítva, azonban nem minden esetben nyújtja a legjobb megoldást.

Ugyanitt megadhatjuk az alapértelmezett utazási módot, ami lehet gyaloglás, valamint vezetés.

Bár az egyszerűség kedvéért igyekeztem sok írás helyett szimbólumokat tenni a gombokra, de az applikációban található kevés szövegnek a nyelvét szintén itt lehet beállítani.

A térképpel, navigációval kapcsolatos paraméterek módosítására is van lehetőség, mint példaul az alapértelmezett nagyítás, a pozíció lekérésének gyakorisága, a térképre való automatikus ránagyítás és ennek centralizálásának a gyakorisága navigáció közben, valamint a megtett út kirajzolásának a mennyisége.

%%%%%%%%%%%%%%%%%%%%%%%%%%%%%%%%%%%%%%%%%%%%%%%%%%%%%%%%%%%%%%%%%%%%%%%
\section{Adatbázis}\label{sec:ALAP:adatelem}

%%%%%%%%%%%%%%%%%%%%%%%%%%%%%%%%%%%%%%%%%%%%%%%%%%%%%%%%%%%%%%%%%%%%%%%
\section{Osztályok}\label{sec:ALAP:adatelem}

%%%%%%%%%%%%%%%%%%%%%%%%%%%%%%%%%%%%%%%%%%%%%%%%%%%%%%%%%%%%%%%%%%%%%%%
\section{Felmerülő problémák}\label{sec:ALAP:adatelem}

