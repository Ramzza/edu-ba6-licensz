%!TEX root = minta_dolgozat.tex
%%%%%%%%%%%%%%%%%%%%%%%%%%%%%%%%%%%%%%%%%%%%%%%%%%%%%%%%%%%%%%%%%%%%%%%
\chapter{Technológiák}\label{ch:ALAP}
%%%%%%%%%%%%%%%%%%%%%%%%%%%%%%%%%%%%%%%%%%%%%%%%%%%%%%%%%%%%%%%%%%%%%%%

\begin{osszefoglal}
	Ebben a fejezetben, a projekt elkészítése során felhasznált technológiákat ismertetem.
	
\end{osszefoglal}

%%%%%%%%%%%%%%%%%%%%%%%%%%%%%%%%%%%%%%%%%%%%%%%%%%%%%%%%%%%%%%%%%%%%%%%
\section{Android}\label{sec:ALAP:ml}

Az Android egy nyílt forráskódú, Linux kernel alapú többfelhasználós operációs rendszer, ahol minden applikáció egy külön felhasználó. Hivatalos nyelvei a Java és a Kotlin. Alapértelmezetten, a rendszer minden applikációnak kioszt egy egyedi Linux felhasználói ID-t (az ID-t csak a rendszer használja, és ismeretlen az applikáció számára). Az operációs rendszer úgy osztja ki a hozzáfárési jogokat az applikáció állományai számára, hogy csak az a felhasználói ID férjen hozzájuk, amivel az adott applikáció rendelkezik. Minden folyamat (process) rendelkezik a saját virtuális gépével, tehát minden applikáció kódja egymástól teljesen izoláltan fut. Alapértelmezetten minden applikáció a saját Linux folyamatában fut. Az Android operációs rendszer elindítja a folyamatot, amikor az applikáció valamelyik komponensének szüksége van rá, majd leállítja azt, mikor többé már nincs rá szükség, vagy ha a rendszernek mermóriát kell lefoglalnia, más applikációk számára. Az Android operációs rendszer “a legkevesebb kiváltság elvét” (“the principle of least privilige”) alkalmazza, tehát alapértelmezetten, minden applikációnak csak azokhoz a komponensekhez van hozzáférése, amik szükségesek a feladatának elvégzéséhez, és semmi egyébhez.

Egy android applikációnak négy fajta komponense lehet: Activity, Service, Broadcast receiver, Content provider. Ezek közül az Activity szolgál a felhasználóval történő interakció eszközéül, ez ugyanis egy felhasználói felülettel rendelkező képernyő. Activity-k, Service-k és Broadcast receiver-ek egy aszinkron üzenet által aktiválódnak, amit Intent-nek (szándék) nevezünk.

Az Activity egy osztály (class), amely a logikát tartalmazza, a felhasználói felületért azonban egy ehhez tartozó XML állomány felel, ami a különböző UI elemeket (gombok, szövegdobozok, konténerek) tartalmazza.

Az AndroidManifest.xml egy konfigurációs állomány, ami a projekt gyökér könyvtárában található. Itt vannak deklarálva az applikáció komponensei, valamint a szükséges hozzáférési jogokat is itt kell feltünteni.



%%%%%%%%%%%%%%%%%%%%%%%%%%%%%%%%%%%%%%%%%%%%%%%%%%%%%%%%%%%%%%%%%%%%%%%
\section{Gradle}\label{sec:ALAP:adatelem}

A Gradle egy nyílt forráskódú projektépítő eszköz (build automation tool). A Groovy, vagy Kotlin nyelvű scriptekbe, megadhatjuk a projektünk külső függöségeit (például API-k, library-k), melyeket letölti, majd lekompillálja és lefordítja a forráskódot.

%%%%%%%%%%%%%%%%%%%%%%%%%%%%%%%%%%%%%%%%%%%%%%%%%%%%%%%%%%%%%%%%%%%%%%%
\section{Firebase}\label{sec:ALAP:szerkeszt}

A felhasználók adatainak, beállításainak tárolására, kezelésére a Firestore nevű no-sql (.json) alapú adatbázist használtam, mely a CRUD%
\footnote{ %
	create, read, update, delete
}  %
 műveletekhez is biztosít metódusokat. Logikai éptőelemei a kollekciók (collection) és a dokumentumok (document). Az előbbi tartalmazhat dokumentumokat, míg az utóbbi alkollekciókat, vagy magukat az adatokat. Az adatok kulcs, érték párok, az értékek lehetnek számok, karakterláncok, tömbök, geopontok vagy akár sajátos osztályok. Amennyiben sajátos osztály akarunk használni, annak rendelkeznie kell egy publikus konstruktorral, melynek nincsenek paraméterei, valamint az attribútumokhoz kell tartozzon egy-egy publikus „getter”%
 \footnote{ %
 	egy paraméter nélküli metódus, mely egy attribútumot térít vissza - példa: String getName()
 }  %
 metódus. Az adatbázisban található adatokhoz való hozzáférést egy szabállyal kell megadni, amelyet a szerver leellenőriz a CRUD műveletek végrehajtása esetén.

A felhasználók menedzselésére, mint a regisztráció, bejelentkezés, e-mail cím aktiválása, elfelejtett jelszó visszaállítása, a Firebase Authentication szolgáltatást használtam.


%%%%%%%%%%%%%%%%%%%%%%%%%%%%%%%%%%%%%%%%%%%%%%%%%%%%%%%%%%%%%%%%%%%%%%%
\section{Maps API}\label{sec:ALAP:szerkeszt}

A csomópontok közti távolságok mátrixának lekérdezésére a Distance Matrix API-t, valamint két csomópont közötti útvonal meghatározására a Directions API-t használtam.
