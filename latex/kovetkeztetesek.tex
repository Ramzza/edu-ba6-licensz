%!TEX root = GRoutes.tex
%%%%%%%%%%%%%%%%%%%%%%%%%%%%%%%%%%%%%%%%%%%%%%%%%%%%%%%%%%%%%%%%%%%%%%%
\chapter{Következtetések}\label{ch:ALAP}
%%%%%%%%%%%%%%%%%%%%%%%%%%%%%%%%%%%%%%%%%%%%%%%%%%%%%%%%%%%%%%%%%%%%%%%

A tudomány jelenlegi állása szerint nem létezik olyan algoritmus, amely polinomiális komplexitással rendelkezik, és meg tudja oldani egzakt módon az utazó ügynök problémáját. A gyakorlati felhasználói esetekben, azonban nem is biztos, hogy mindig erre van szükség. Az alkalmazásokat adaptívnak kell készítenünk, és a felhasználónak meg kell adjuk a lehetőséget, hogy a saját felhasználási igényeihez mérten testre tudja szabni azokat. Így tehát ez a projekt egy ötletet adhat azoknak, akik bizonyos helyzetekben egy, míg bizonyos helyzetekben más megoldási módszereket alkalmaznának. Az egzakt algoritmusoknál a bemenő adatok növekedésével párhuzamosan, a végrehajtási idő exponenciálisan növekszik. A mohó algoritmusok ezzel ellentétben nagy mennyiségű bemeneti adatra is nagyon gyorsan képesek meghatározni egy megoldást, azonban az nem biztos, hogy ez az optimális megoldás lesz. Azt viszont csak a felhasználó tudja, hogy milyen célra, vagy mikor milyen célra szeretné használni az alkalmazást. Így tehát a döntés lehetőségét az ő kezébe helyeztük.

A későbbiekben szeretném az alkalmazást egy újabb funkcionalitással, valamint platformmal kibővíteni. A csoportok menüpont alatt lehetőség nyílik arra, hogy a felhasználók egymásnak tervezzenek útvonalakat, megosszák ezeket, valamint arra is, hogy cégen belül az irodából lehessen frissíteni a terepen dolgozó kolléga további állomásait. Ehhez szeretnék majd egy web-es felületet is készíteni, hogy az irodából egyszerűbben, kényelmesebben lehessen nagyobb mennyiségű adattal dolgozni. A személyreszabhatóság növelésének érdekében több kritérium megadására is lehetőség lesz, mint például az adott helyszínekhez köthető nyitvatartási időintervallum\cite{gt_problem}\cite{gt_and_appl}\cite{gt_alg_india}\cite{comp_tsp}
