%!TEX root = GRoutes.tex
%%%%%%%%%%%%%%%%%%%%%%%%%%%%%%%%%%%%%%%%%%%%%%%%%%%%%%%%%%%%%%%%%%%%%%%
\chapter{Gráf problémák}\label{ch:ALAP}
%%%%%%%%%%%%%%%%%%%%%%%%%%%%%%%%%%%%%%%%%%%%%%%%%%%%%%%%%%%%%%%%%%%%%%%

\begin{osszefoglal}
	Ebben a fejezetben a különböző gráfelméleti problémákat fogom röviden ismertetni.
	
\end{osszefoglal}

%%%%%%%%%%%%%%%%%%%%%%%%%%%%%%%%%%%%%%%%%%%%%%%%%%%%%%%%%%%%%%%%%%%%%%%
\section{Gráfok leszámlálása}\label{sec:ALAP:adatelem}

%%%%%%%%%%%%%%%%%%%%%%%%%%%%%%%%%%%%%%%%%%%%%%%%%%%%%%%%%%%%%%%%%%%%%%%
\section{Részgráf izomorfizmus probléma}\label{sec:ALAP:adatelem}

A részgráf izomorfizmus egy komputacionális probléma, adottak a \(H\) és \(G\) gráfok, és el kell dönteni, hogy létezik-e \(G\)-nek olyan részgráfja, amely izomorf \(H\)-val. Ennek bizonyítása egy NP-teljes feladat.

%%%%%%%%%%%%%%%%%%%%%%%%%%%%%%%%%%%%%%%%%%%%%%%%%%%%%%%%%%%%%%%%%%%%%%%
\section{Gráfok színezése}\label{sec:ALAP:adatelem}

Egy gráf színezése azt jelenti, hogy a csúcsaihoz (vagy az éleihez) színeket rendelünk (legtöbbször egy számmal reprezentáljuk), úgy, hogy két bármely két szomszédos csúcs (vagy él) különböző színű legyen.

%%%%%%%%%%%%%%%%%%%%%%%%%%%%%%%%%%%%%%%%%%%%%%%%%%%%%%%%%%%%%%%%%%%%%%%
\section{Útproblémák}\label{sec:ALAP:adatelem}

\subsection{Hamilton-kör probléma}

A Hamilton-kör egy gráfban, egy olyan kör, mely áthalad a gráf összes csúcsán. Mivel egy kör, minden csúcson egyszer fog áthaladni, kivéve az elsőt, ami egyben az utolsó is. Egy gráfot Hamilton-gráfnak nevezünk, ha tartalmaz Hamilton-kört.

Egy egyszerű gráf \(G = (X,E)\), \(n \leq 3\) csúccsal Hamilton gráf, ha \(d(x) + d(y)  \geq n\), minden nem szomszédos \(x,y \in X\) csúcsra.

Ebből kifolyólag egy gráfban a Hamilton-kör létezése elméleti szemszögből nem egy egyszerú probléma, de gyakorlati szemszögből sem, mivel NP%
\footnote{ %
	nem determinisztikusan polinomiális
}  %
-teljes. Az utazó ügynök problémája magába foglalja a Hamilton-kör létezését a gráfban.

\subsection{Minimális feszítőfa}

Egy \(fa\) egy olyan gráf, amely összefüggő és nem tartalmaz köröket. Bármely összefüggő \(G\) gráfban a feszítőfa \(G\)-nek egy olyan részgráfja, amely fa, és tartalmazza \(G\) összes csúcsát. A minimális feszítőfát meghatározhatjuk Prim algoritmusa segítségével.

Prim algoritmusa: kezdjük bármelyik csúccsal, madj válasszuk ki a hozzá tartozó legkisebb súllyal rendelkező élet. Minden iterációban kiválasztjuk az eddig vizsgált csúcsokhoz tartozó élek közül a legkisebb súllzal rendelkező élet, majd hozzáadjuk az eddig kapott fához az eddig nem vizsgált csúcspontjával együtt. Ezt addig folyatjuk, míg az összes csúcsot megvizsgáltuk.

\subsection{Kínaipostás-probléma}

Egy \(G\) gráfban, találjuk meg a legrövidebb zárt sétát, amely minden élen legalább egyszer halad át. Egy optimális megoldást találni úgy irányított, mint irányítatlan gráfban NP-teljes.

\subsection{Königsbergi hidak problémája}

Königsberg városában volt hét híd, amely összekötött két szigetet a város két partjával és egymással. A lakosok megpróbáltak úgy sétálni, hogy minden hídon egyszer és csakis egyszer haladjanak át, és érjenek vissza a kezdőpontba. Ez soha senkinek nem sikerült Euler magyarázta meg egy 1736-ban írt cikkben, hogy ez miért nem lehetséges.

Egy Euler-út egy gráfban egy olyan út, amely minden élet magába foglal, egyszer és csakis egyszer. Egy Euler-út zárt, ha ugyanabban a csúcsban végződik, mint ahonnan elindul, egyébként nyíltnak nevezzük.

Egy gráf akkor és csakis akkor tartalmaz zárt Euler-utat, ha minden csúcshoz páros számú él tartozik, valamint minden él ugyanahoz a komponenshez tartozik. Egy gráf akkor és csakis akkor tartalmaz nyílt Euler-utat, ha pontosan két csúcshoz tartozik páratlan számú él, és minden él egyazon komponenshez tartozik.

\subsection{Legrövidebb út probléma}

Dijkstra algoritmusa megoldja a legrövidebb út problémat és visszatéríti a legrövidebb utat a kiinduló csúcsból az összes többi csúcsba. Negatív súlyú élek esetén nem működik. Az algoritmus komplexitása \(O(n^2)\), ahol \(n\) a csúcsok száma.


